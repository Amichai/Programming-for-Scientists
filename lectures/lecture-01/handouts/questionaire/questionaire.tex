\documentclass[article,twoside]{memoir}
\usepackage[latin9]{inputenc}
\usepackage{psfrag}
\usepackage{hyperref}
\usepackage{eepic}
\usepackage{color}
\usepackage{amsmath}
\usepackage{amssymb}
\usepackage{bm}
\usepackage{amsfonts}
\usepackage{amsthm}
\usepackage{sublabel}
\usepackage[vlined,linesnumbered,ruled]{algorithm2e}
%\usepackage[]{pgf}
\usepackage[]{graphicx}
\usepackage{subfigure}

\graphicspath{{../figures/other/}{../figures/generated/}}


\DeclareMathOperator*{\argmax}{arg\,max}
\DeclareMathOperator*{\argmin}{arg\,min}

\newcommand{\pd}[2]{\frac{\partial#1}{\partial#2}}
\newcommand{\pdd}[2]{\frac{\partial^2 #1}{\partial #2^2}} 
\newcommand{\pdpd}[3]{\frac{\partial^2 #1}{\partial #2 \partial #3}} 
\newcommand{\B}{\ensuremath{\mathbb{B}}}
\newcommand*{\Pa}{\ensuremath{\text{\upshape\textbf{Pa}}}}
\newcommand*{\Dkl}{\ensuremath{D_{\text{\textsc{kl}}}}}
\newcommand*{\encspace}{\quad}

\newtheorem{theorem}{Theorem}
\newtheorem{lemma}{Lemma}
\newtheorem{definition}{Definition}

\newcommand*{\slantfrac}[2]{\hbox{$\raisebox{-.4ex}{$\,^#1$}\!/_#2$}}
\newcommand*{\onehalf}{\ensuremath{
{\frac{1}{2}}%
}}
\newcommand*{\dx}{\,dx}
\newcommand*{\dt}{\,dt}

\newcommand*{\Expected}{\ensuremath{\mathbb{E}}}
\newcommand*{\Var}{\ensuremath{\text{Var}}}
\newcommand*{\Reals}{\ensuremath{\mathbb{R}}}
\newcommand*{\Binary}{\ensuremath{\mathbb{B}}}
\newcommand*{\discr}{\mbox{discr}}
\newcommand*{\from}{\leftarrow}

\newcommand*{\indicator}[1]{\hspace{1pt}[\hspace{-.4em}[\hspace{3pt} #1 \hspace{3pt}]\hspace{-.4em}]\hspace{2pt}}
\newcommand*{\Assign}{\ensuremath\,:=\,}
\newcommand*{\vect}[1]{\ensuremath{\bm{#1}}}
\newcommand*{\textvalue}[1]{\mbox{\textsl{#1}}}
\newcommand*{\bigO}{\mathcal{O}}
\newcommand*{\MI}{\mbox{MI}}

\DeclareMathOperator{\goodness}{score}
\DeclareMathOperator{\powerset}{Pow}

\title{Student Survey}
\author{Programming for Scientists}

\bibliographystyle{alpha}
%\bibliographystyle{plain}
\pagestyle{Ruled}

\aliaspagestyle{chapter}{Ruled}
\makeatletter
\if@twoside
	\makeoddhead{Ruled}{}{}{\scshape\rightmark}
	\makeevenhead{Ruled}{\scshape\leftmark}{}{}
	\makeevenfoot{Ruled}{\pagenumberfont\thepage}{}{}
	\makeoddfoot{Ruled}{}{}{\pagenumberfont\thepage}
	% Put section number on top
	\def\sectionmark#1{\markright{#1 (\thesection)}}
	\def\chaptermark#1{\markboth{#1}{#1}}
	\renewcommand*{\bibmark}{\markboth{\bibname}{\bibname}} % I don't like empty headings!
\else

	% Put section number on top
	\def\sectionmark#1{\markright{#1 (\thesection)}}
	\def\chaptermark#1{\markright{#1 (\thechapter)}}

	\makeoddhead{Ruled}{}{}{\scshape\rightmark}
	\makeevenhead{Ruled}{}{}{\scshape\rightmark}
	\makeevenfoot{Ruled}{}{}{\pagenumberfont\thepage}
	\makeoddfoot{Ruled}{}{}{\pagenumberfont\thepage}
	\renewcommand*{\bibmark}{\markright{\bibname}} % I don't like empty headings!


\fi
\makeatother

%Change fonts: Page number sans-serif (much cleaner than the roman version)
\def\pagenumberfont{\sffamily}
% Change fonts: Section headers Sans-Serif:
\setsecheadstyle{\Large\sffamily\raggedright}
\setsubsecheadstyle{\large\sffamily\raggedright}
\setsubsubsecheadstyle{\normalsize\sffamily\raggedright}

% Title
\pretitle{\LARGE\sffamily}
\posttitle{\par\vspace{4ex}}
\preauthor{\large\sffamily\hspace{1cm}}
\postauthor{\par\vspace{3ex}}
\predate{\small\sffamily\hspace{1cm}Last Updated on: }
\postdate{\par\vspace{2cm}}
\copypagestyle{title}{plain}
\makeoddfoot{title}{}{}{}
\makeevenfoot{title}{}{}{}

\renewcommand{\abstractnamefont}{\sffamily}
\renewcommand{\abstracttextfont}{}
\renewcommand{\absnamepos}{flushleft}


\makechapterstyle{mestrado}{% Originally ``AlexanderGrebenkov'', adapted
\renewcommand{\chapterheadstart}{\goodbreak\vspace*{\beforechapskip}\medskip}
\renewcommand{\chapnamefont}{\normalfont\Large\scshape}
\renewcommand{\chapnumfont}{\normalfont\Large\scshape}
\renewcommand{\chaptitlefont}{\normalfont\Large\scshape}
\renewcommand{\printchaptername}{}
\renewcommand{\chapternamenum}{}
\renewcommand{\printchapternum}{\normalfont\Large\scshape\S\thechapter}
\renewcommand{\afterchapternum}{\hspace{1em}}
\renewcommand{\afterchaptertitle}{\par\nobreak\vspace{-.9em}\moveright 6pt\vbox to 1pt{\hrule width .64\textwidth}\nobreak\vskip\afterchapskip\nobreak}
}
\chapterstyle{mestrado}

\SetCommentSty{textsl}



\newcommand*{\fillunderscore}{\textunderscore \textunderscore \textunderscore \textunderscore \textunderscore \textunderscore \textunderscore \textunderscore \textunderscore \textunderscore \textunderscore \textunderscore \textunderscore \textunderscore \textunderscore \textunderscore \textunderscore \textunderscore \textunderscore \textunderscore \textunderscore}
\newcommand*{\checkbox}{$\square$}
\newcommand*{\yesno}{\hfill\mbox{yes\checkbox~no\checkbox}}
\newcommand{\header}[1]{\textsl{#1}}
\begin{document}
\maketitle
\begin{abstract}
Please fill out this survey. It will help steer the class in the right direction
\end{abstract}

\chapter{About You}

\header{This is just general information about you.}

Name: \fillunderscore\\
Major/Programme: \fillunderscore\\
I am taking this class for credit: \yesno.

\chapter{Motivation}

Why are you taking this class?
\begin{itemize}
\item To learn to program \checkbox.
\item To learn to program \emph{better} \checkbox.
\item To learn Python \checkbox.
\item My advisor told me to take it \checkbox.
\end{itemize}

Others (write as many as you can think of): \fillunderscore\\
\fillunderscore\\
\fillunderscore\\
\fillunderscore

\chapter{Background}

\section{Programming Languages}

\begin{tabular}{lccccc}
 & Never heard of it & Heard of it & Used it at most a couple of times & Used it a lot & Expert\\
Matlab  &\checkbox &\checkbox &\checkbox &\checkbox &\checkbox \\
Java    &\checkbox &\checkbox &\checkbox &\checkbox &\checkbox \\
C       &\checkbox &\checkbox &\checkbox &\checkbox &\checkbox \\
C++     &\checkbox &\checkbox &\checkbox &\checkbox &\checkbox \\
Python  &\checkbox &\checkbox &\checkbox &\checkbox &\checkbox \\
R       &\checkbox &\checkbox &\checkbox &\checkbox &\checkbox \\
Fortran &\checkbox &\checkbox &\checkbox &\checkbox &\checkbox \\
Perl    &\checkbox &\checkbox &\checkbox &\checkbox &\checkbox \\
Lisp    &\checkbox &\checkbox &\checkbox &\checkbox &\checkbox \\
Scheme  &\checkbox &\checkbox &\checkbox &\checkbox &\checkbox \\
\end{tabular}

\section{Programming Concepts}
\begin{tabular}{lcccccc}
 \toprule
 & Never heard & Heard & Vague & I know what & I could write \\
 & of it & of it & Idea& it means& my dissertation \\
 & & & & & about it\\
 \midrule
Object-oriented Programming            &\checkbox &\checkbox &\checkbox &\checkbox &\checkbox \\
Big O Notation (e.g., $\bigO(\log N)$) &\checkbox &\checkbox &\checkbox &\checkbox &\checkbox \\
Space-time trade-off                   &\checkbox &\checkbox &\checkbox &\checkbox &\checkbox \\
Linked Lists                           &\checkbox &\checkbox &\checkbox &\checkbox &\checkbox \\
\end{tabular}

\section{Things you know}

\begin{itemize}
\item I know what open-source means \yesno.
\item I know the difference between the GPL and the BSD license \yesno.
\item I know the difference between ``amortised constant time'' and ``constant time'' \yesno.
\end{itemize}


\section{Software usage}

I use the following operating system(s) regularly:
\begin{itemize}
\item Windows XP \checkbox
\item Windows Vista \checkbox
\item Linux \checkbox (Any particular distribution? \fillunderscore)
\item Mac OS X \checkbox
\item Other Unix variant \checkbox (Which? \fillunderscore)
\end{itemize}

I am able to use the following operating system(s)

\begin{itemize}
\item Windows XP \checkbox
\item Windows Vista \checkbox
\item Mac OS X \checkbox
\item Non-mac Unix (for example, Linux) \checkbox
\end{itemize}

\chapter{Interests}

\section{Special Topics}

\header{The last section of the course has some room for modification (see the syllabus). Please tell me what you would like to hear about.}

\begin{tabular}{lc}
Image Processing & \checkbox\\
Interfacing with Other Languages & \checkbox\\
Concurrent (Parallel) Programming & \checkbox\\
\end{tabular}

Something else: \fillunderscore.

\chapter{Comments}

\header{Please add any other comments you fill might be helpful.}

\vspace*{10em}

\end{document}
