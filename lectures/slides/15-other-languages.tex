\documentclass{beamer}
\usetheme{CMU}

\usepackage{pgf,pgfarrows,pgfnodes,pgfautomata,pgfheaps,pgfshade}
\usepackage{amsmath,amssymb}
\usepackage[utf8]{inputenc}
\usepackage{colortbl}
\usepackage[english]{babel}
\usepackage{booktabs}
\usepackage{slpython}
\usepackage{underscore}

\author{Luís Pedro Coelho}
\institute{Programming for Scientists}

\graphicspath{{figures/}{figures/generated/}{images/}}

\newcommand*{\code}[1]{\textsl{#1}}


\title{Interfacing With Other Languages}
\begin{document}
\frame{\maketitle}

\begin{frame}[fragile]
\frametitle{Python: Language \& Implementation}
Python is a language, but it's also a program.
\end{frame}

\begin{frame}[fragile]
\frametitle{Definition}

A \alert{successful language} is a language with more than one implementation.
\note{One of which is an open-source language.}

\end{frame}

\begin{frame}[fragile]
\frametitle{CPython}
The main Python implementation is \alert{CPython}.
\end{frame}

\begin{frame}[fragile]
\frametitle{Other Implementations}
\begin{itemize}
\item Stackless Python (C)
\item Jython (Java)
\item IronPython (.NET)
\item PyPy (Python in Python)
\item Pyjamas (Python to Javascript!)
\item \ldots
\end{itemize}
\end{frame}

\begin{frame}[fragile]
\frametitle{Writing Extensions}
Writing extensions to CPython can be done\\
in C(++) or similar languages (Fortran).
\end{frame}

\begin{frame}[fragile]
\frametitle{Writing an Extension Example}

\begin{python}
def countchar(string,c):
    '''
    C = countchar(string,ch)

    Counts the number of instances of character ch
    in string string. Returns an integer.
    '''
    res = 0
    for ch in string:
        if c == ch:
            res += 1
    return res
\end{python}
\end{frame}


\begin{frame}[fragile]
\frametitle{Writing A Python Extension}

\begin{ccode}
PyObject* countchar(PyObject* self, PyObject* args) {
    const char* string;
    const char ch;
    if (!PyArg_ParseTuple(args,"sc",&string,&ch))
        return NULL;
    int res = 0;
    while (*string) {
        if (ch == *string) ++res;
        ++string;
    }
    return Py_BuildValue("i", res);
}
\end{ccode}
\end{frame}

\begin{frame}[fragile]
\frametitle{Writing a Python Extension}
\begin{enumerate}
\item Get the arguments into \alert{C variables}
\item Do computation
\item Put the arguments into \alert{Python variables}
\item Return
\end{enumerate}
\end{frame}

\begin{frame}[fragile]
\frametitle{Get the Arguments}

\begin{block}{PyArg\_ParseTuple}
Like \alert{printf} (or \alert{scanf}), takes a format string.

Powerful, but fragile.
\end{block}
\end{frame}

\begin{frame}[fragile]
\frametitle{Telling Python About Your Function}

\begin{ccode}
static PyMethodDef methods[] = {
    {"countchar",  countchar, METH_VARARGS,
            "countchar(str,ch)\n"
            "Counts the number of ..."},
    {NULL, NULL, 0, NULL} /* Sentinel */
};

PyMODINIT_FUNC initcountchar(void)
{
    (void) Py_InitModule("countchar", methods);
}
\end{ccode}

\end{frame}

\begin{frame}[fragile]

\begin{block}{Embedding C++}
Trivial.

Just remember to add a couple of
\alert{extern "C"} here and there.
\end{block}
\end{frame}

\begin{frame}[fragile]
\frametitle{C++ Example}
\begin{ccode}

extern "C" {
    #include <Python.h>
}

...
const char* module_doc = 
    "countchar module.\n\n"
    "This is a great module.\n";

extern "C"
void initcountchar()
  {
    (void)Py_InitModule3("countchar", methods, module_doc);
  }
\end{ccode}
\end{frame}

\begin{frame}[fragile]
\frametitle{Structuring}

\begin{block}{Two Layers}
\begin{enumerate}
\item \alert{Python}: Massage the arguments
\item \alert{C(++)}: Do computation
\item \alert{Python}: Massage results
\end{enumerate}
\end{block}
\end{frame}

\begin{frame}[fragile]
\frametitle{Example}

\begin{python}
import _countchar

def countchar(s,ch):
    '''
    countchar(str,ch)

    Counts the ...
    '''
    if s is None:
        return 0
    return _countchar(str,ch)
\end{python}
\end{frame}

\begin{frame}[fragile]
\frametitle{Alternatives}

\begin{block}{SWIG}
Does most of what you've seen automatically.
\end{block}

\begin{block}{Boost.Python}
Allows mixed language applications (Python/C++).
\end{block}

\end{frame}

\begin{frame}[fragile]
\frametitle{SWIG}
\begin{block}{Pros}
\begin{itemize}
\item Easy to use.
\item Widely used.
\item Well supported by build tools.
\end{itemize}
\end{block}

\begin{block}{Cons}
\begin{itemize}
\item Error messages are not Pythonic.
\item Leads to undocumented functions.
\item \alert{C++ only:} template support lacking.
\note{To be fair to swig, templates are a really different model for the tool to worry about.}
\end{itemize}
\end{block}
\end{frame}

\begin{frame}[fragile]
\frametitle{Boost.Python}
\begin{block}{Pros}
\begin{itemize}
\item Tight integration of C++/Python
\item Amazing Technology
\end{itemize}
\end{block}

\begin{block}{Cons}
\begin{itemize}
\item Hard to use
\end{itemize}
\end{block}
\end{frame}

\begin{frame}[fragile]
\frametitle{Summary}
\begin{enumerate}
\item Use \alert{swig} as your first pass.
\item Write your own for control.
\item Use Boost.python for very large projects.
\end{enumerate}
\end{frame}

\begin{frame}[fragile]
\frametitle{Fortran}
There exists a tool, called \alert{f2py} which is\\
similar to swig for Fortran.
\end{frame}



\end{document}
