\documentclass{beamer}
\usetheme{CMU}

\usepackage{pgf,pgfarrows,pgfnodes,pgfautomata,pgfheaps,pgfshade}
\usepackage{amsmath,amssymb}
\usepackage[utf8]{inputenc}
\usepackage{colortbl}
\usepackage[english]{babel}
\usepackage{booktabs}
\usepackage{slpython}
\usepackage{underscore}

\author{Luís Pedro Coelho}
\institute{Programming for Scientists}

\graphicspath{{figures/}{figures/generated/}{images/}}

\newcommand*{\code}[1]{\textsl{#1}}


\title{Python Review}
\begin{document}
\frame{\maketitle}

\begin{frame}[fragile]
\frametitle{What's The Bug?}

\begin{python}
def mean(values):
    i = 0
    S = 0.0
    while i <= len(values):
        S += values[i]
        i += 1
    return S/len(values)
\end{python}
\end{frame}
\begin{frame}[fragile]
\frametitle{Corrected Version}

\begin{python}
def mean(values):
    '''
    M = mean(values)

    Return the mean value of a sequence.
    '''
    if not values:
        raise ValueError(
            'mean: Mean of empty set not defined')
    S = 0.0
    for val in values:
        S += val
    return S/float(len(values))
\end{python}
\note{
    Added docstring, exception,floating-point conversion.
}
\end{frame}

\begin{frame}[fragile]
Write a function that takes another function $f$ and a number $N$ and computes

\begin{equation*}
\sum_{i=0}^{N} f(N)^2.
\end{equation*}

\end{frame}

\begin{frame}[fragile]
\frametitle{What's the Bug?}

\begin{python}
def cull_herd(bacteria):
    for i in xrange(len(bacteria)):
        if random.random() < bacteria[i].P_dead():
            del bacteria[i]
\end{python}
\end{frame}

\begin{frame}[fragile]
\frametitle{Corrected}
\begin{python}
def cull_herd(bacteria):
    i = 0
    while i < len(bacteria):
        if random.random() < bacteria[i].P_dead():
            del bacteria[i]
        else:
            i += 1
\end{python}
\end{frame}

\begin{frame}[fragile]

\begin{block}{hello.py}
\begin{python}
print 'Hello World'
\end{python}
\end{block}

\begin{block}{main.py}
\begin{python}
import hello
import hello
\end{python}
\end{block}

What does \texttt{python main.py} print out?

\begin{columns}
\column{.3\textwidth}
(a)\\
Hello World\\
Hello World

\column{.3\textwidth}
(b)\\
\textit{nothing}

\column{.3\textwidth}
(c)\\
Hello World
\end{columns}

\end{frame}

\begin{frame}[fragile]
\begin{python}
class Bacterium(object):
    def __init__(self):
        print 'Bacterium'
    ...

class EvolveBacterium(Bacterium):
    def __init__(self):
        print 'EvolveBacterium'
    ...

bac = EvolveBacterium()
\end{python}

\begin{columns}
\column{.25\textwidth}
(a)\\
\verb+Bacterium+

\column{.25\textwidth}
(b)\\
\verb+EvolveBacterium+\\

\column{.25\textwidth}
(c)\\
\verb+Bacterium+\\
\verb+EvolveBacterium+\\

\column{.25\textwidth}
(d)\\
\verb+EvolveBacterium+\\
\verb+Bacterium+\\
\end{columns}

\end{frame}


\begin{frame}[fragile]
\begin{python}
A = range(20)
B = A[10:20]
del B[0]
del B[0]
del B[0]
print len(A)
\end{python}

\begin{columns}
\column{.25\textwidth}
(a)\\
20
\column{.25\textwidth}
(b)\\
17
\column{.25\textwidth}
(c)\\
16
\column{.25\textwidth}
(d)\\
19
\end{columns}
\end{frame}

\begin{frame}[fragile]
\begin{python}
A = range(20)
B = A[10:20]
B[0] = -1
print A[10]
\end{python}

\begin{columns}
\column{.25\textwidth}
(a)\\
-1
\column{.25\textwidth}
(b)\\
10
\column{.25\textwidth}
(c)\\
9
\column{.25\textwidth}
(d)\\
11
\end{columns}
\end{frame}

\begin{frame}[fragile]
\begin{python}
def swap(x,y):
    x, y = y,x

a,b = 0,1
swap(a,b)
print a, b
\end{python}

\begin{flushright}
(From: \textit{How to Think Like a Computer Scientist}
\end{flushright}

\begin{columns}
\column{.5\textwidth}
(a)\\
0,1
\column{.5\textwidth}
(b)\\
1,0
\end{columns}
\end{frame}


\begin{frame}[fragile]
\begin{python}
def swap(A):
    A[0],A[1] = A[1],A[0]

A = [0,1]
swap(A)
print A[0],A[1]
\end{python}

\begin{columns}
\column{.5\textwidth}
(a)\\
0,1
\column{.5\textwidth}
(b)\\
1,0
\end{columns}
\end{frame}

\begin{frame}[fragile]
Write a \alert{generator} that iterates through a list in \alert{reverse} order:

\begin{python}
for val in reversed(range(4)):
    print val
\end{python}
should print

\begin{python}
3
2
1
0
\end{python}
\end{frame}

\begin{frame}[fragile]
\frametitle{Trick Question}

\begin{python}
A = [0,1,2]
B = A
A += [3]
print B
\end{python}

What does this print?

\begin{columns}
\column{.4\textwidth}
(a)\\
\verb+[0,1,2]+
\column{.4\textwidth}
(b)\\
\verb+[0,1,2,3]+
\end{columns}
\end{frame}

\begin{frame}[fragile]
\begin{python}
def mystery(*args):
    if not args:
        yield ()
    else:
        lst = args[0]
        rest = args[1:]
        for val in lst:
            for tp in mystery(*rest):
                yield (val,) + tp
\end{python}

\pause

\textbf{Hint:} Consider

\begin{python}
for T in mystery(range(2),range(4)):
    print T
\end{python}
\end{frame}

\begin{frame}[fragile]
\begin{python}
def f(x):
    print 'f(%s)' % x
    return x + 2
def g(x):
    print 'g(%s)' % x
    return log(f(x-1))
def h(x):
    print 'h(%s)' % x
    return g(x)-g(x-1)

print h(0)
\end{python}

\begin{columns}
\column{.25\textwidth}
(b)\\
h(0)\\
g(0)\\
OverflowError: math range error
\column{.25\textwidth}
(b)\\
h(0)\\
g(0)\\
f(-1)\\
OverflowError: math range error
\column{.25\textwidth}
(c)\\
h(0)\\
g(0)\\
f(-1)\\
g(-1)\\
f(-1)\\
OverflowError: math range error
\column{.25\textwidth}
(d)\\
h(0)\\
g(0)\\
f(-1)\\
g(-1)\\
f(-1)
\end{columns}
\end{frame}

\end{document}
