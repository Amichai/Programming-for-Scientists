\documentclass{beamer}
\usetheme{CMU}

\usepackage{pgf,pgfarrows,pgfnodes,pgfautomata,pgfheaps,pgfshade}
\usepackage{amsmath,amssymb}
\usepackage[utf8]{inputenc}
\usepackage{colortbl}
\usepackage[english]{babel}
\usepackage{booktabs}
\usepackage{slpython}
\usepackage{underscore}

\author{Luís Pedro Coelho}
\institute{Programming for Scientists}

\graphicspath{{figures/}{figures/generated/}{images/}}

\newcommand*{\code}[1]{\textsl{#1}}


\title{Introduction to Python}
\begin{document}


\begin{frame}[fragile]
\frametitle{Installing Python}

Let's digress for a moment discussing the language\ldots
\end{frame}


\begin{frame}[fragile]
\frametitle{Python Language History}

\begin{block}{History}
Python was started in the late 80's. It was intended to be both \alert{easy to teach} and \alert{industrial strength}.

It is (has always been) open-source and has become one of the most widely used languages (top 10).
\end{block}
\end{frame}

\begin{frame}[fragile]
\frametitle{Python Language}

Unlike \textit{Matlab}, Python does not come as a single programme.

This is a cultural difference.

\end{frame}

\begin{frame}[fragile]
\frametitle{Example of Python IDEs}

\begin{block}{Integrated Development Environments}
\begin{itemize}
\item Eclipse
\item Stani (Stan's Python editor)
\item DrPython
\item Komodo Edit
\item Unix world: ipython + (vim or emacs or nano) + command-line
\item \ldots
\end{itemize}
\end{block}

Links for the above are on the webpage under ``notes for lab session 1''.
\end{frame}

\begin{frame}
\frametitle{Python Versions}

\begin{block}{Python Versions}
\begin{itemize}
\item The current version of Python is \alert{2.6} (October 2008).
\item This class assumes you have 2.5 or 2.6.
\item There are some small differences, but only one difference is important (we'll get to that).
\end{itemize}
\end{block}

\begin{block}{Python 3.0}
Python 3.0 changes many more things. We'll, for the most part, ignore it.
\end{block}
\end{frame}

\begin{frame}[fragile]

\begin{block}{Running Python}
\begin{enumerate}
\item From a file
\item Interactively
\end{enumerate}
\end{block}

\end{frame}

\begin{frame}[fragile]
\frametitle{Back to the Python Language}
Let's look at the language itself.
\end{frame}

\begin{frame}[fragile]
\frametitle{Python Example}

\begin{python}
print "Hello World"
\end{python}
\end{frame}

\frame{\frametitle{Task}
\begin{block}{Average}
Compute the average of the following numbers:
\begin{enumerate}
\item 10
\item 7
\item 22
\item 14
\item 17
\end{enumerate}
\end{block}

}

\begin{frame}[fragile]
\frametitle{Python example}

\begin{python}
numbers = [10, 7, 22, 14, 17]

sum = 0
n = 0
for val in numbers:
    sum = sum + val
    n = n + 1
return sum / n
\end{python}

\end{frame}

\begin{frame}[fragile]
``Python is executable pseudo-code.''

---Python lore (often attributed to Bruce Eckel)

\end{frame}


\frame{\frametitle{Programming Basics}

A 5~minute introduction to programming\ldots
}

\begin{frame}[fragile]
\frametitle{Python Types}
\begin{block}{Basic Types}
\begin{itemize}
\item Numbers (integers and floating point)
\item Strings
\item Lists and tuples
\item Dictionaries
\end{itemize}
\end{block}
\end{frame}


\begin{frame}[fragile]
\frametitle{Python Types: Numbers I: Integers}
\begin{python}
A = 1
B = 2
C = 3
print A+B*C
\end{python}

Outputs \alert{7}.
\end{frame}

\begin{frame}[fragile]
\frametitle{Python Types: Numbers II: Floats}
\begin{python}
A = 1.2
B = 2.4
C = 3.6
print A + B*C
\end{python}

Outputs \alert{9.84}.
\end{frame}

\begin{frame}[fragile]
\frametitle{Python Types: Numbers III: Integers \& Floats}
\begin{python}
A = 2
B = 2.5
C = 4.4
print A + B*C
\end{python}

Outputs \alert{22.0}.
\end{frame}

\begin{frame}[fragile]
\frametitle{Python Types: Strings}

\begin{python}
first = 'John'
last = "Doe"
full = first + " " + last

print full
\end{python}

\pause
Outputs \alert{John Doe}.
\end{frame}

\begin{frame}[fragile]
\frametitle{Python Types: String Rules}

\begin{block}{What is a String Literal}
\begin{itemize}
\item Short string literals are delimited by (") or (').
\item Short string literals are one line only.
\item Special characters are input using escape sequences.\\
        ($\backslash$n for newline,\ldots)
\end{itemize}
\end{block}

\begin{python}
multiple = 'Let  me go back and face the peril.\\nNo, it\'s too perilous.'
alternative = "Let  me go back and face the peril.\\nNo, it's too perilous."
\end{python}
\end{frame}

\begin{frame}[fragile]
\frametitle{Python Types: Long Strings}

We can input a long string using triple quotes (''' or """) as delimiters.

\begin{python}
long = '''Tell me, is love
Still a popular suggestion
Or merely an obsolete art?

Forgive me, for asking,
This simple question,
I'm unfamiliar with his heart.'''
\end{python}
\end{frame}

\begin{frame}[fragile]
\frametitle{Python Types: Lists}
\begin{python}
to_read = ['Crime and Punishment','Brothers Karamazov']

print "The first book I am going to read is", to_read[0]
print "The second book is", to_read[1]

\only<2>{to_read.sort()}
\only<2>{print 'I prefer alphabetical order:', to_read[0], to_read[1]}
\end{python}

Notice that list indices start at \alert{0}!

\end{frame}

\begin{frame}[fragile]
\frametitle{List Indexing}
\begin{block}{Indices Start at Zero}
\begin{itemize}
\item First element is denoted \code{list[0]}, last element is \code{list[N-1]}.
\item Think of them as \alert{offsets}.
\end{itemize}
\end{block}
\end{frame}


\begin{frame}[fragile]
\frametitle{Python Types: Lists}

\begin{python}
mixed = ['Banana',100,['Another','List'],[]]
print len(mixed)
\end{python}

\end{frame}

\begin{frame}[fragile]
\frametitle{Python Types: Dictionaries}
\begin{python}
emails = { 'Luis' : 'lpc@cmu.edu',
           'Mark' : 'mark@cmu.edu' }
print 'Luis's email is', emails['Luis']

emails['Rita'] = 'rita@cmu.edu'
\end{python}

\end{frame}

\begin{frame}[fragile]
\frametitle{Python Control Structures}

\begin{python}
student = 'Rita'
average = gradeavg(student)
if average > 0.7:
    print student, 'passed!'
    print 'Congratulations!!'
else:
    print student, 'failed. Sorry.'
\end{python}
\end{frame}

\begin{frame}[fragile]
\frametitle{Python Blocks}

Unlike almost all other modern programming languages,\\
Python uses \alert{indentation} to delimit blocks!

\begin{python}
if <condition>:
    statement 1
    statement 2
    statement 3
next statement
\end{python}

\end{frame}

\begin{frame}[fragile]
\frametitle{Adhere to Convention}
\begin{block}{Principle}
\begin{enumerate}
\item Learn about the conventions of the technology you're using.
\item Adhere to them.
\end{enumerate}
\end{block}

Use 4 spaces to indent.
\end{frame}

\begin{frame}[fragile]
\frametitle{Conditionals}


\begin{block}{Examples}
\begin{itemize}
\item \python{x == y}
\item \python{x != y}
\item \python{x < y}
\item \python{x < y < z}
\item \python{x in lst}
\item \python{x not in lst}
\end{itemize}
\end{block}

\end{frame}

\begin{frame}[fragile]
\frametitle{Nested Blocks}

\begin{python}
if <condition 1>:
    do something
    if condition 2>:
        nested block
    else:
        nested else block
elif <condition 1b>:
    do something
\end{python}
\end{frame}

\begin{frame}[fragile]
\frametitle{For loop}

\begin{python}
emails = { 'Luis' : 'lpc@cmu.edu',
           'Mark' : 'mark@cmu.edu',
           'Rita' : 'rsmith@pitt.edu' }
students = ['Rita','Sarah','Mark']
for st in students:
    if st not in emails.keys():
        print 'We need to ask', st, 'for her email'
\end{python}
\end{frame}

\begin{frame}[fragile]
\frametitle{While Loop}

\begin{python}
i = 0
while i < len(students):
    if is_enrolled(students[i]):
        i += 1
    else:
        del students[i]
\end{python}
\end{frame}

\begin{frame}[fragile]
\frametitle{Other Loopy Stuff}

\begin{python}
for i in range(5):
    print i
\end{python}

prints

\begin{verbatim}
0
1
2
3
4
\end{verbatim}
\end{frame}

\begin{frame}[fragile]
\frametitle{Range}
\begin{python}
R = range(5)
R2 = [0,1,2,3,4]
\end{python}

R and R2 are identical lists.
\end{frame}

\begin{frame}[fragile]
\frametitle{Xrange: Faster looping}
\begin{python}
for i in xrange(5):
    print i
\end{python}

prints

\begin{verbatim}
0
1
2
3
4
\end{verbatim}
\end{frame}

\begin{frame}[fragile]
\frametitle{Break}
\begin{python}
rita_enrolled = False
for st in students:
    if st == 'Rita':
        rita_enrolled = True
        break
\end{python}

\end{frame}

\begin{frame}[fragile]
\frametitle{Continue}
\begin{python}
for st in students:
    if turned_in_homework(st):
        continue
    print '%s did not turn in his homework on time' % st
    <statement>
    <statement>
    <statement>
\end{python}

Notice the string formatting on the last line!
\end{frame}

\begin{frame}[fragile]
\frametitle{String Formatting}
\begin{python}
"string with %s placeholders %s" % (arg1,arg2)
\end{python}

\begin{itemize}
\item The \code{\%s} placeholders are replaced by the passed arguments with the \code{\%} operator.
\item Use \code{\%\%} to get a \%.
\end{itemize}

\end{frame}

\begin{frame}[fragile]
\frametitle{Conditions \& Booleans}
\begin{block}{Booleans}
\begin{itemize}
\item Just two values: \code{True} and \code{False}.
\item Comparisons return booleans (e.g., \code{$x < 2$})
\end{itemize}
\end{block}

\begin{block}{Conditions}
\begin{itemize}
\item When evaluating a condition, the condition is converted to a boolean:
\item Many things are converted to \code{False}:
\begin{enumerate}
\item \code{$[]$} (the empty list)
\item \code{$\{ \}$} (the empty dictionary)
\item \code{""} (the empty string)
\item \code{0} or \code{0.0} (the value zero)
\item \ldots
\end{enumerate}
\item Everything else is \code{True} or not convertible to boolean.
\end{itemize}
\end{block}
\end{frame}

\begin{frame}[fragile]
\frametitle{Conditions Example}

\begin{python}
A = []
B = [1,2]
C = 2
D = 0

if A:
    print 'A is true'
if B:
    print 'B is true'
if C:
    print 'C is true'
if D:
    print 'D is true'
\end{python}
\end{frame}

\begin{frame}[fragile]
\frametitle{Numbers}
\begin{block}{Two Types of Numbers}
\begin{enumerate}
\item Integers
\item Floating-point
\end{enumerate}
\end{block}

\begin{block}{Operations}
\begin{enumerate}
\item Unary Minus: \code{ -x}
\item Addition: \code{x + y}
\item Subtraction: \code{x - y}
\item Multiplication: \code{x * y}
\item Exponentiation: \code{x ** y}
\end{enumerate}
\end{block}
\end{frame}

\begin{frame}[fragile]
\frametitle{Division}
\begin{block}{Division}
What is 9 divided by 3?

What is 10 divided by 3?
\end{block}

\begin{block}{Two types of division}
\begin{enumerate}
\item Floating-point division: \code{x / y} (in Python 2.6)
\item Integer division: \code{x // y}
\end{enumerate}
\end{block}

\begin{block}{Python 2.5 vs.\ 2.6}
\begin{itemize}
\item This is the one important difference between the two versions of Python:
\begin{enumerate}
\item In Python 2.5, \code{x / y} means \alert{integer division} by default.
\item In Python 2.6, it means \alert{floating-point division}.
\end{enumerate}
\item We can get the 2.6 behaviour by adding \code{from __future__ import division} at the top of your file
\end{itemize}
\end{block}
\end{frame}

\begin{frame}[fragile]
\frametitle{Importing}

\begin{python}
A = 10
B = 3
print A / B
\end{python}

\begin{python}
from __future__ import division
A = 10
B = 3
print A / B
\end{python}
\end{frame}


\begin{frame}[fragile]
\frametitle{Functions}
\begin{python}
def double(x):
    '''
    y = double(x)

    Returns the double of x
    '''
    return 2*x
\end{python}
\end{frame}

\begin{frame}[fragile]
\frametitle{Functions}
\begin{python}
A=4
print double(A)
print double(2.3)
print double(double(A))
\end{python}
\end{frame}


\end{document}
