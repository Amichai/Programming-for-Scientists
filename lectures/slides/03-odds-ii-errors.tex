\documentclass{beamer}
\usetheme{CMU}

\usepackage{pgf,pgfarrows,pgfnodes,pgfautomata,pgfheaps,pgfshade}
\usepackage{amsmath,amssymb}
\usepackage[utf8]{inputenc}
\usepackage{colortbl}
\usepackage[english]{babel}
\usepackage{booktabs}
\usepackage{slpython}
\usepackage{underscore}

\author{Luís Pedro Coelho}
\institute{Programming for Scientists}

\graphicspath{{figures/}{figures/generated/}{images/}}

\newcommand*{\code}[1]{\textsl{#1}}


\title{Python's Odds and Ends (II)\\Error Handling}
\begin{document}
\frame{\maketitle}

\section{Odds \& Ends (II)}

\section{Libraries}

\begin{frame}[fragile]
\frametitle{Modules \& Libraries}
\begin{python}
import math

math.exp(1)
\end{python}
\end{frame}

\begin{frame}[fragile]
\frametitle{Namespaces}
Namespaces are where names live.
\end{frame}

\begin{frame}[fragile]

\begin{block}{bacteria.py}
\begin{python}
...
def simulate(...):
    '''...'''
    ...
...
\end{python}
\end{block}

\begin{block}{script.py}
\begin{python}
import bacteria

population = [bacteria.Bacterium(...) ...]
bacteria.simulate(...)
\end{python}
\end{block}

\end{frame}

\begin{frame}[fragile]
\frametitle{Importing (II)}

\begin{python}
import bacteria
simulate = bacteria.simulate
Bacterium = bacteria.Bacterium
\end{python}

\pause

\begin{python}
from bacteria import simulate, Bacterium
\end{python}

\end{frame}

\begin{frame}[fragile]
\frametitle{Importing (III)}

\begin{python}
import bacteria
import bacteria
import bacteria
import bacteria
import bacteria
import bacteria
\end{python}
\end{frame}

\begin{frame}[fragile]
\frametitle{Import All}

\begin{python}
from bacteria import *
\end{python}

\end{frame}

\begin{frame}[fragile]
\frametitle{Import As}

\begin{python}
import bacteria
bac = bacteria

bac.simulate(...)
\end{python}

\lstinline{bac} is another name for \lstinline{bacteria} (modules are objects too!)

\pause

\begin{python}
import bacteria as bac
\end{python}
\end{frame}

\begin{frame}[fragile]
\begin{python}
def enumerate(iterable):
    '''...'''
    i = 0
    for val in iterable:
        yield i,val
        i += 1
\end{python}

\begin{python}
buildings = ['Scaiffe','Wean','Mellon']
for i,build in enumerate(buildings):
    print i, build
\end{python}
\end{frame}

\begin{frame}[fragile]
\frametitle{Zip}

\begin{python}
names = ['Rita','Luis','Sabah']
grades = ['A','B','A']

for g,n in zip(names,grades):
    print 'Student %s had grade %s' % (g,n)
\end{python}

\end{frame}

\begin{frame}[fragile]
\frametitle{File Reading}

\begin{python}
for line in file('filename.txt'):
    print line
\end{python}

\end{frame}

\begin{frame}[fragile]
\frametitle{File Objects}

\begin{python}
input = file('myfile.data')
for line in input:
    ...
\end{python}

\begin{python}
output = file('myoutput.txt','w')
print >>output, 'Hello World\n'
output.close()
\end{python}
\end{frame}

\begin{frame}[fragile]
\frametitle{File Objects}
\begin{python}
input = file('myfile.data','r')
for i in xrange(5):
    # First five lines are comments
    input.readline()

sixth = input.readline()
rest = input.readlines()
\end{python}
\end{frame}


\begin{frame}[fragile]
\frametitle{Errors}
Errors happen.
\end{frame}

\begin{frame}[fragile]
\frametitle{Errors}

\begin{itemize}
\item File doesn't exist.
\item File is misformatted.
\item Some data is not appropriate for the situation.
\item You have a bug in your code.
\item \ldots
\end{itemize}

\end{frame}

\begin{frame}[fragile]
\frametitle{Handling Errors}

What should a function that reads a file do if the file doesn't exist?

What should the log() function return for negative numbers?
\end{frame}

\begin{frame}[fragile]
\frametitle{Exceptions}

\begin{block}{Exceptions}
Report errors for higher up.
\end{block}

\end{frame}

\begin{frame}[fragile]
\frametitle{Call Stack}
\begin{python}
def f(x):
    return log(x)**2

def g(x):
    y = f(x)
    return y+1

def h(x):
    return g(x+1) + g(4*x)

print h(0)
\end{python}
\end{frame}

\begin{frame}[fragile]
\frametitle{Exceptions}

\begin{python}
def log(x):
    if x <= 0.:
        raise ValueError(
            'log: argument must be greater than zero')
    ...
\end{python}
\end{frame}

\begin{frame}[fragile]
\frametitle{Try-Except}

\begin{python}
try:
    h(0)
except:
    print 'Ooops'
\end{python}
\end{frame}

\begin{frame}[fragile]
\frametitle{Try-Except}

\begin{python}
try:
    <line 1>
    <line 2>
    <line 3>
except:
    <line 1>
    <line 2>
\end{python}
\end{frame}

\begin{frame}[fragile]
\frametitle{Exceptions}

\begin{block}{Exceptions}
\begin{itemize}
\item Exceptions are \alert{objects}.
\item Exceptions have \alert{type} and \alert{values}.
\end{itemize}
\end{block}
\end{frame}

\begin{frame}[fragile]
\frametitle{Exception Hierarchy}

(Nothing here, folks, look at the blackboard)

\end{frame}

\begin{frame}[fragile]
\frametitle{Exception Handling}

\begin{block}{Exception Handling: Error Handling}
\begin{python}
try:
    <code 1>
    <code 2>
    <code 3>
except IOError, exc:
    print 'I/O Error', exc
except:
    print 'Unspecified error'
finally:
    <code 1>
    <code 2>
\end{python}
\end{block}
\end{frame}

\begin{frame}[fragile]

\begin{python}
def f(x):
    if x <= 0.:
        raise ValueError(
            'f: argument must be greater than zero')
    return sqrt(x)+2

def g(x):
    y = f(x)
    print (y > 2) 

try:
    g(1)
    g(-1)
except:
    print 'Exception'
\end{python}

This outputs:

\begin{columns}
\column{.25\textwidth}
(a)\par
True\\True
\column{.25\textwidth}
(b)\par
True\\False
\column{.25\textwidth}
(c)\par
False\\Exception
\column{.25\textwidth}
(d)\par
True\\Exception
\end{columns}

\end{frame}


\end{document}
