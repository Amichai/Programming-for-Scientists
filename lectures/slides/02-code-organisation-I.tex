\documentclass{beamer}
\usetheme{CMU}

\usepackage{pgf,pgfarrows,pgfnodes,pgfautomata,pgfheaps,pgfshade}
\usepackage{amsmath,amssymb}
\usepackage[utf8]{inputenc}
\usepackage{colortbl}
\usepackage[english]{babel}
\usepackage{booktabs}
\usepackage{slpython}
\usepackage{underscore}

\author{Luís Pedro Coelho}
\institute{Programming for Scientists}

\graphicspath{{figures/}{figures/generated/}{images/}}

\newcommand*{\code}[1]{\textsl{#1}}


\title{Code Organisation}
\begin{document}

\frame{\maketitle}

\begin{frame}[fragile]
\frametitle{Procedural Programming}
\alert{Procedural programming} means organising programs around functions.

\alert{Object-oriented programming} means organising programs around objects.
\end{frame}

\begin{frame}[fragile] 
\frametitle{Object Oriented Programming}

\begin{block}{OOP}
\begin{description}
\item Aggregation: organise functions into classes.
\item Encapsulation: hide information inside methods.
\item Polymorphism: re-use code for multiple types.
\end{description}
\end{block}
\end{frame}

\begin{frame}[fragile] 
\frametitle{User-Defined Types}

\begin{block}{Built-in Types}
\begin{enumerate}
\item lists
\item dictionaries
\item strings
\item \ldots
\end{enumerate}
\end{block}
\end{frame}

\begin{frame}[fragile] 
\frametitle{Type}
\begin{block}{What's a Type}
\begin{enumerate}
\item A domain of values
\item A set of methods (functions)
\end{enumerate}
\end{block}

\end{frame}

\begin{frame}[fragile] 
\frametitle{Examples of Types}

\begin{block}{List}
\begin{enumerate}
\item Domain: lists
\item Functions: \code{L.append(e),L.insert(idx,e), \ldots}
\item Operators: \code{L[0], 'Rita' in L}
\end{enumerate}
\end{block}

\pause
\begin{block}{Integer}
\begin{enumerate}
\item Domain: $\dots,-2, 1, 0, 1, 2, \dots$
\item Operators: \code{A + B},\ldots
\end{enumerate}
\end{block}
\end{frame}

\begin{frame}[fragile] 
\frametitle{User-defined Types}

Object-oriented programming languages allow one to define one's own types.

\end{frame}

\begin{frame}[fragile] 
\frametitle{Why User-defined Types}

\begin{block}{Why Define Your Own Types?}
\begin{enumerate}
\item Elegance: pre-defined types shouldn't be special.
\item Aggregation: closely related data and function should be tied together.
\item Encapsulation: implementation details can be hidden.
\end{enumerate}
\end{block}

\end{frame}

\begin{frame}[fragile] 
\frametitle{Motivating Example}
\begin{block}{Simple Population Simulation}
\begin{enumerate}
\item Our world is characterised by a single environmental value $e$.
\item We have ``animals'' that are in the world.
\item Each animal has two characteristics: adaptation $a$ and mutation rate $\sigma$.
\item The smaller the difference $|a-e|$, the better an animal is adapted to the world.
\item When an animal reproduces, its offspring has adaptation $a + \cal{N}(0,\sigma)$
\item At each iteration:
\begin{enumerate}
\item Animals die with a probability given by $\exp(\beta|a-e|)$
\item Animals that survive, reproduce.
\end{enumerate}
\end{enumerate}
\end{block}

\end{frame}

\begin{frame}[fragile] 
\frametitle{Animal World}
\begin{block}{Animal Class}
We define an Animal class, with two values:
\begin{enumerate}
\item adaptation: its current adaptation value
\item sigma: its variability parameter
\end{enumerate}
and two methods:
\begin{enumerate}
\item \code{dies()}: make a stochastic decision on whether the animal dies
\item \code{reproduce()}: make a new animal, derived from current one
\end{enumerate}
\end{block}

\end{frame}

\begin{frame}[fragile] 
\frametitle{Using our Animals}
\begin{python}
animals = [Animal() for i in xrange(nr_inital_animals)]
environ = 1.0
for i in xrange(max_iters):
    ai = 0
    while ai < len(animals):
        if animals[i].dies(environ): del animals[i]
        else: ai += 1
    N=len(animals)
    if N == 0:
        print 'Our population died out!'
        break
    for ai in xrange(len(ai)):
        animals.append(animals[i].reproduce())
    if N >= max_animals:
        print 'Our population is out of control'
        break
\end{python}
\end{frame}

\begin{frame}[fragile] 
\frametitle{Using our Animals}

\begin{python}
\ldots
DeltaAdaptation = [math.abs(y-a.adaptation) 
                    for a in animals]
Sigmas = [a.sigma for a in animals]
hist(Sigmas)
\end{python}

\end{frame}

\begin{frame}[fragile] 
\frametitle{Classes As Logical Units}
\begin{block}{Class}
A class aggregates data and functions that belong together.
\end{block}
\end{frame}

\begin{frame}[fragile] 
\frametitle{Animal Interface}

\begin{block}{Interface}
\begin{enumerate}
\item Constructor: Takes the initial adaptation value and sigma.
\item \code{died(environ)}: Make a stochastic decision on whether the Animal dies in the given environment.
\item \code{reproduce()}: Return a new Animal.
\item \code{adaptation}: Current adaptation.
\item \code{sigma}: Current sigma.
\end{enumerate}
\end{block}
\end{frame}

\begin{frame}[fragile] 
\frametitle{Animals}
\begin{block}{Sigma Varying}
Suppose we want to vary sigma from one generation to the next?
\end{block}

The \alert{code that uses Animals} need not change!
\end{frame}
\begin{frame}[fragile] 
\frametitle{Animals}

\end{frame}

\begin{frame}[fragile] 
\frametitle{Polymorphism}

\begin{block}{Type Polymorphism}
Code is \alert{polymorphic} if it can use different types without change
\end{block}
\end{frame}

\begin{frame}[fragile] 
\frametitle{Animal Class}

\begin{python}
class Animal(object):
    '''
    Animal class

    Implements an animal
    ...
    '''
    def __init__(self,adaptation,sigma):
        self.adaptation = adaptation
        self.sigma = sigma
    
    def P_dead(self,environ):
        '''
        prob = animal.P_dead(environ)

        ...
        '''
        return L*math.exp(-abs(self.adaptation-environ))
    ...
\end{python}
\end{frame}

\end{document}
