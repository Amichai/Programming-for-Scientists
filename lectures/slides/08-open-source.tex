\documentclass{beamer}
\usetheme{CMU}

\usepackage{pgf,pgfarrows,pgfnodes,pgfautomata,pgfheaps,pgfshade}
\usepackage{amsmath,amssymb}
\usepackage[utf8]{inputenc}
\usepackage{colortbl}
\usepackage[english]{babel}
\usepackage{booktabs}
\usepackage{slpython}
\usepackage{underscore}

\author{Luís Pedro Coelho}
\institute{Programming for Scientists}

\graphicspath{{figures/}{figures/generated/}{images/}}

\newcommand*{\code}[1]{\textsl{#1}}


\title{Distribution of Software}
\begin{document}
\frame{\maketitle}

\begin{frame}[fragile]
\frametitle{Open Source}
What is open-source?
\end{frame}

\begin{frame}[fragile]
\frametitle{Intellectual Property}
Wikipedia-level introduction to intellectual property:
\begin{enumerate}
\item copyright
\item patent \note{software is patentable in the US, but not in other juridisctions [EU]}
\item trademark
\end{enumerate}
\end{frame}

\begin{frame}[fragile]
\frametitle{Licenses}
A license is a \alert{loosening} of the default ``All Rights Reserved'' framework.
\note{
    Note that the code is owned either by he who writes it or by their lab/university/\ldots
}
\end{frame}

\begin{frame}[fragile]
\frametitle{Licenses}
\begin{enumerate}
\item Public-domain
\item BSD or MIT licenses
\item GPL (copy-left)
\item Author License
\item \ldots (10~million others)
\end{enumerate}
\end{frame}

\begin{frame}[fragile]
\frametitle{Public Domain}

You can release the code into the public domain.
\end{frame}

\begin{frame}[fragile]
\frametitle{Non-Code Open Source}

\begin{itemize}
\item Creative Commons.
\item GFDL
\item \ldots
\end{itemize}
\note{
    GFDL is the GNU Free Documentation License, on which Wikipedia is based.
}
\end{frame}

\begin{frame}[fragile]
\frametitle{Creative Commons}

\begin{block}{Restrictions}
\begin{itemize}
\item \alert{Attribution} [By] you need to state the source
\item \alert{Non-Commercial} [NC] you cannot use it for commercial purposes
\item \alert{Non-Derivative} [ND] you may not modify the work
\item \alert{Share-Alike} [SA] you must license your derivative works similarly 
\end{itemize}
\note{
    These are all a bit 'undefined' and mostly un-tested by case law.

    What counts as non-commercial, for example?
}
\end{block}

Mix \& match to get a license, e.g., By-NC-SA.\note{
    This is the license of the class.
}
\end{frame}

\end{document}
