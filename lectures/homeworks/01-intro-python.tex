\documentclass[article,twoside]{memoir}
\usepackage[latin9]{inputenc}
\usepackage{eepic}
\usepackage{color}
\usepackage{amsmath}
\usepackage{amssymb}
\usepackage{bm}
\usepackage{amsfonts}
\usepackage{amsthm}
\usepackage[vlined,linesnumbered,ruled]{algorithm2e}
%\usepackage[]{pgf}
\usepackage[]{graphicx}
\usepackage{python}

\graphicspath{{../figures/other/}{../figures/generated/}}


\DeclareMathOperator*{\argmax}{arg\,max}
\DeclareMathOperator*{\argmin}{arg\,min}

\newcommand{\pd}[2]{\frac{\partial#1}{\partial#2}}
\newcommand{\pdd}[2]{\frac{\partial^2 #1}{\partial #2^2}} 
\newcommand{\pdpd}[3]{\frac{\partial^2 #1}{\partial #2 \partial #3}} 
\newcommand{\B}{\ensuremath{\mathbb{B}}}
\newcommand*{\Pa}{\ensuremath{\text{\upshape\textbf{Pa}}}}
\newcommand*{\Dkl}{\ensuremath{D_{\text{\textsc{kl}}}}}
\newcommand*{\encspace}{\quad}

\newtheorem{theorem}{Theorem}
\newtheorem{lemma}{Lemma}
\newtheorem{definition}{Definition}

\newcommand*{\slantfrac}[2]{\hbox{$\raisebox{-.4ex}{$\,^#1$}\!/_#2$}}
\newcommand*{\dx}{\,dx}
\newcommand*{\dt}{\,dt}

\newcommand*{\Expected}{\ensuremath{\mathbb{E}}}
\newcommand*{\Var}{\ensuremath{\text{Var}}}
\newcommand*{\Reals}{\ensuremath{\mathbb{R}}}
\newcommand*{\Binary}{\ensuremath{\mathbb{B}}}
\newcommand*{\discr}{\mbox{discr}}
\newcommand*{\from}{\leftarrow}

\newcommand*{\indicator}[1]{\hspace{1pt}[\hspace{-.4em}[\hspace{3pt} #1 \hspace{3pt}]\hspace{-.4em}]\hspace{2pt}}
\newcommand*{\Assign}{\ensuremath\,:=\,}
\newcommand*{\vect}[1]{\ensuremath{\bm{#1}}}
\newcommand*{\textvalue}[1]{\mbox{\textsl{#1}}}
\newcommand*{\bigO}{\mathcal{O}}
\newcommand*{\MI}{\mbox{MI}}

\DeclareMathOperator{\goodness}{score}
\DeclareMathOperator{\powerset}{Pow}

\title{Student Survey}
\author{Programming for Scientists}

\bibliographystyle{alpha}
%\bibliographystyle{plain}
\pagestyle{plain}

\aliaspagestyle{chapter}{Ruled}

%Change fonts: Page number sans-serif (much cleaner than the roman version)
\def\pagenumberfont{\sffamily}
% Change fonts: Section headers Sans-Serif:
\setsecheadstyle{\Large\sffamily\raggedright}
\setsubsecheadstyle{\large\sffamily\raggedright}
\setsubsubsecheadstyle{\normalsize\sffamily\raggedright}

% Title
\pretitle{\LARGE\sffamily}
\posttitle{\par\vspace{4ex}}
\preauthor{\large\sffamily\hspace{1cm}}
\postauthor{\par\vspace{3ex}}
\predate{\small\sffamily\hspace{1cm}Due on: }
\postdate{\par\vspace{2cm}}
\copypagestyle{title}{plain}
\makeoddfoot{title}{}{}{}
\makeevenfoot{title}{}{}{}

\renewcommand{\abstractnamefont}{\sffamily}
\renewcommand{\abstracttextfont}{}
\renewcommand{\absnamepos}{flushleft}


\makechapterstyle{mestrado}{% Originally ``AlexanderGrebenkov'', adapted
\renewcommand{\chapterheadstart}{\goodbreak\vspace*{\beforechapskip}\medskip}
\renewcommand{\chapnamefont}{\normalfont\Large\scshape}
\renewcommand{\chapnumfont}{\normalfont\Large\scshape}
\renewcommand{\chaptitlefont}{\normalfont\Large\scshape}
\renewcommand{\printchaptername}{}
\renewcommand{\chapternamenum}{}
\renewcommand{\printchapternum}{\normalfont\Large\scshape\S\thechapter}
\renewcommand{\afterchapternum}{\hspace{1em}}
\renewcommand{\afterchaptertitle}{\par\nobreak\vspace{-.9em}\moveright 6pt\vbox to 1pt{\hrule width .64\textwidth}\nobreak\vskip\afterchapskip\nobreak}
}
\chapterstyle{mestrado}

\SetCommentSty{textsl}

\newcounter{questioncnt}
\newcommand*{\question}{\addtocounter{questioncnt}{1}\noindent\textbf{Question \Roman{questioncnt}: }}


\author{Programming for Scientists}
\title{Homework 1}
\date{Jan 27}
\begin{document}
\maketitle

\chapter{Questions}

\question
Which function returns the number of elements in a list?

\question
How do you access the first element of a list? How do you access the last?

\question
Write a function that implements the factorial function (i.e., $n! = \Pi_{i=1}^n i$). Don't forget the documentation!


\question
Alice and Bob were doing some \textit{pair programming}\footnote{Pair programming is when two programmers write code together. Normally, one will sit at the computer typing with the other will look over his shoulder to make sure he isn't making any mistakes. After a while, they switch.

You might consider pair programming to help with assignments in this class.} and Alice wrote the following code:

\begin{python}
amount = 100
interest = 5
amount *= (1. + interest)

print 'Value after interest:', amount 
\end{python}

Bob argued that this would never work. ``In Python, numbers are immutable. In the line \lstinline{amount *= (1. + interest)}, you are changing the value of \lstinline{amount}. However, when they try it out, it works as Alice expected. Where is the mistake in Bob's reasoning? Are numbers mutable after all?

\question
What is the difference between the following two code examples:

A)

\begin{python}
A = [1, 2, 3]
B = [1, 2, 3]
\end{python}

B)

\begin{python}
A = [1, 2, 3]
B = A
\end{python}

Write a small (should be 2 or 3 lines of code) piece of code that behaves differently if you insert it after the two segments above.

\chapter{Programming Assignment}

You are considering whether to enroll in a couple of different saving plans. You put some money in now and deduct some initial fee, you put in some money every month, interest accrues each month and is added to the pot (at the end of the month, just before you add in more). How much money would you have after 5 years under each plan?

\begin{tabular}{lcccc}
\toprule
Plan Name & Fee & Initial Amount & Monthly Amount & Interest \\
\midrule
Low-fee & 0 & 0 & 100 & .2\% \\
Low-interest & 20 & 100 & 50 & .02\% \\
High-initial & 100 & 1000 & 50 & .2\% \\
High-monthly & 100 & 0 & 250 & .2\% \\
High-interest & 100 & 0 & 100 & .4\% \\
\bottomrule
\end{tabular}

\end{document}

