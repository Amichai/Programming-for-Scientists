\documentclass[article,twoside]{memoir}
\usepackage[latin9]{inputenc}
\usepackage{eepic}
\usepackage{color}
\usepackage{amsmath}
\usepackage{amssymb}
\usepackage{bm}
\usepackage{amsfonts}
\usepackage{amsthm}
\usepackage[vlined,linesnumbered,ruled]{algorithm2e}
%\usepackage[]{pgf}
\usepackage[]{graphicx}
\usepackage{python}

\graphicspath{{../figures/other/}{../figures/generated/}}


\DeclareMathOperator*{\argmax}{arg\,max}
\DeclareMathOperator*{\argmin}{arg\,min}

\newcommand{\pd}[2]{\frac{\partial#1}{\partial#2}}
\newcommand{\pdd}[2]{\frac{\partial^2 #1}{\partial #2^2}} 
\newcommand{\pdpd}[3]{\frac{\partial^2 #1}{\partial #2 \partial #3}} 
\newcommand{\B}{\ensuremath{\mathbb{B}}}
\newcommand*{\Pa}{\ensuremath{\text{\upshape\textbf{Pa}}}}
\newcommand*{\Dkl}{\ensuremath{D_{\text{\textsc{kl}}}}}
\newcommand*{\encspace}{\quad}

\newtheorem{theorem}{Theorem}
\newtheorem{lemma}{Lemma}
\newtheorem{definition}{Definition}

\newcommand*{\slantfrac}[2]{\hbox{$\raisebox{-.4ex}{$\,^#1$}\!/_#2$}}
\newcommand*{\dx}{\,dx}
\newcommand*{\dt}{\,dt}

\newcommand*{\Expected}{\ensuremath{\mathbb{E}}}
\newcommand*{\Var}{\ensuremath{\text{Var}}}
\newcommand*{\Reals}{\ensuremath{\mathbb{R}}}
\newcommand*{\Binary}{\ensuremath{\mathbb{B}}}
\newcommand*{\discr}{\mbox{discr}}
\newcommand*{\from}{\leftarrow}

\newcommand*{\indicator}[1]{\hspace{1pt}[\hspace{-.4em}[\hspace{3pt} #1 \hspace{3pt}]\hspace{-.4em}]\hspace{2pt}}
\newcommand*{\Assign}{\ensuremath\,:=\,}
\newcommand*{\vect}[1]{\ensuremath{\bm{#1}}}
\newcommand*{\textvalue}[1]{\mbox{\textsl{#1}}}
\newcommand*{\bigO}{\mathcal{O}}
\newcommand*{\MI}{\mbox{MI}}

\DeclareMathOperator{\goodness}{score}
\DeclareMathOperator{\powerset}{Pow}

\title{Student Survey}
\author{Programming for Scientists}

\bibliographystyle{alpha}
%\bibliographystyle{plain}
\pagestyle{plain}

\aliaspagestyle{chapter}{Ruled}

%Change fonts: Page number sans-serif (much cleaner than the roman version)
\def\pagenumberfont{\sffamily}
% Change fonts: Section headers Sans-Serif:
\setsecheadstyle{\Large\sffamily\raggedright}
\setsubsecheadstyle{\large\sffamily\raggedright}
\setsubsubsecheadstyle{\normalsize\sffamily\raggedright}

% Title
\pretitle{\LARGE\sffamily}
\posttitle{\par\vspace{4ex}}
\preauthor{\large\sffamily\hspace{1cm}}
\postauthor{\par\vspace{3ex}}
\predate{\small\sffamily\hspace{1cm}Due on: }
\postdate{\par\vspace{2cm}}
\copypagestyle{title}{plain}
\makeoddfoot{title}{}{}{}
\makeevenfoot{title}{}{}{}

\renewcommand{\abstractnamefont}{\sffamily}
\renewcommand{\abstracttextfont}{}
\renewcommand{\absnamepos}{flushleft}


\makechapterstyle{mestrado}{% Originally ``AlexanderGrebenkov'', adapted
\renewcommand{\chapterheadstart}{\goodbreak\vspace*{\beforechapskip}\medskip}
\renewcommand{\chapnamefont}{\normalfont\Large\scshape}
\renewcommand{\chapnumfont}{\normalfont\Large\scshape}
\renewcommand{\chaptitlefont}{\normalfont\Large\scshape}
\renewcommand{\printchaptername}{}
\renewcommand{\chapternamenum}{}
\renewcommand{\printchapternum}{\normalfont\Large\scshape\S\thechapter}
\renewcommand{\afterchapternum}{\hspace{1em}}
\renewcommand{\afterchaptertitle}{\par\nobreak\vspace{-.9em}\moveright 6pt\vbox to 1pt{\hrule width .64\textwidth}\nobreak\vskip\afterchapskip\nobreak}
}
\chapterstyle{mestrado}

\SetCommentSty{textsl}

\newcounter{questioncnt}
\newcommand*{\question}{\addtocounter{questioncnt}{1}\noindent\textbf{Question \Roman{questioncnt}: }}


\author{Programming for Scientists}
\title{Homework 6}
\date{Mar 3}
\begin{document}
\maketitle

\chapter{Questions}

\question % 1
What is under-flow (when talking about numerical precision)?

\question % 2
What is the smallest (i.e., smallest absolute value) that a 32~bit \textsc{ieee-754} float can represent? (Hint: Wikipedia has the answer to this question).

\question % 3
You will sometimes see the following programming idiom:

\begin{python}
import numpy as np
mystery = np.uint32(-1)
\end{python}

What is the value of mystery? Why would we be interested in this particular value? (Hint: think of its bit representation).

\question % 4
With numpy, we saw that we could easily create an array of 32~bit numbers:

\begin{python}
import numpy as np
A = np.array([1,2,3,4,5],np.int32)
\end{python}

However, it is also possible to simply use a traditional Python list:

\begin{python}
A = [1,2,3,4,5]
\end{python}

This has the advantage of using Python numbers, which are of infinite precision, instead of being limited to 32~bits. Given this obvious disadvantage, why would anyone use the 32~bit array? (There is more than one reason, but you only need to give one).

\question % 5
Consider the following code

\begin{python}
import numpy as np
...
M = A.ptp(0)
\end{python}

If $A$ is of shape $(120,1000)$, what is the resulting shape of $M$? What does the ptp function return?

\chapter{Programming Assignment}

Consider the following approximation to compute an integral:

\begin{equation}
\int_0^{1} f(x)dx \approx \sum_{i = 0}^{999} \frac{f(i/1000)}{1000}.
\end{equation}

\begin{enumerate}
\item Implement two versions of this for integrating $f(x) = x^2$. One should use ``pure Python'' (i.e., you should not require any \lstinline{import}s). Another should be based on \lstinline{numpy} (where you compute all the values $f(i/1000)$ in a single step).
\item Determine which version is faster. (Hint: You might need to run each version a many times to be able to get a meaningful answer.)
\end{enumerate}

\end{document}
