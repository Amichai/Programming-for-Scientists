\documentclass[article,twoside]{memoir}
\usepackage[latin9]{inputenc}
\usepackage{psfrag}
\usepackage{hyperref}
\usepackage{eepic}
\usepackage{color}
\usepackage{amsmath}
\usepackage{amssymb}
\usepackage{bm}
\usepackage{amsfonts}
\usepackage{amsthm}
\usepackage{sublabel}
\usepackage[vlined,linesnumbered,ruled]{algorithm2e}
\usepackage{multirow}
%\usepackage[]{pgf}
\usepackage[]{graphicx}

\graphicspath{{../figures/other/}{../figures/generated/}}


\DeclareMathOperator*{\argmax}{arg\,max}
\DeclareMathOperator*{\argmin}{arg\,min}

\newcommand{\pd}[2]{\frac{\partial#1}{\partial#2}}
\newcommand{\pdd}[2]{\frac{\partial^2 #1}{\partial #2^2}} 
\newcommand{\pdpd}[3]{\frac{\partial^2 #1}{\partial #2 \partial #3}} 
\newcommand{\B}{\ensuremath{\mathbb{B}}}
\newcommand*{\Pa}{\ensuremath{\text{\upshape\textbf{Pa}}}}
\newcommand*{\Dkl}{\ensuremath{D_{\text{\textsc{kl}}}}}
\newcommand*{\encspace}{\quad}

\newtheorem{theorem}{Theorem}
\newtheorem{lemma}{Lemma}
\newtheorem{definition}{Definition}

\newcommand*{\slantfrac}[2]{\hbox{$\raisebox{-.4ex}{$\,^#1$}\!/_#2$}}
\newcommand*{\onehalf}{\ensuremath{
{\frac{1}{2}}%
}}
\newcommand*{\dx}{\,dx}
\newcommand*{\dt}{\,dt}

\newcommand*{\Expected}{\ensuremath{\mathbb{E}}}
\newcommand*{\Var}{\ensuremath{\text{Var}}}
\newcommand*{\Reals}{\ensuremath{\mathbb{R}}}
\newcommand*{\Binary}{\ensuremath{\mathbb{B}}}
\newcommand*{\discr}{\mbox{discr}}
\newcommand*{\from}{\leftarrow}

\newcommand*{\indicator}[1]{\hspace{1pt}[\hspace{-.4em}[\hspace{3pt} #1 \hspace{3pt}]\hspace{-.4em}]\hspace{2pt}}
\newcommand*{\Assign}{\ensuremath\,:=\,}
\newcommand*{\vect}[1]{\ensuremath{\bm{#1}}}
\newcommand*{\textvalue}[1]{\mbox{\textsl{#1}}}
\newcommand*{\bigO}{\mathcal{O}}
\newcommand*{\MI}{\mbox{MI}}

\DeclareMathOperator{\goodness}{score}
\DeclareMathOperator{\powerset}{Pow}

\title{Final Student Survey}
\author{Programming for Scientists}

\bibliographystyle{alpha}
%\bibliographystyle{plain}
\pagestyle{Ruled}

\aliaspagestyle{chapter}{Ruled}
\makeatletter
\if@twoside
	\makeoddhead{Ruled}{}{}{\scshape\rightmark}
	\makeevenhead{Ruled}{\scshape\leftmark}{}{}
	\makeevenfoot{Ruled}{\pagenumberfont\thepage}{}{}
	\makeoddfoot{Ruled}{}{}{\pagenumberfont\thepage}
	% Put section number on top
	\def\sectionmark#1{\markright{#1 (\thesection)}}
	\def\chaptermark#1{\markboth{#1}{#1}}
	\renewcommand*{\bibmark}{\markboth{\bibname}{\bibname}} % I don't like empty headings!
\else

	% Put section number on top
	\def\sectionmark#1{\markright{#1 (\thesection)}}
	\def\chaptermark#1{\markright{#1 (\thechapter)}}

	\makeoddhead{Ruled}{}{}{\scshape\rightmark}
	\makeevenhead{Ruled}{}{}{\scshape\rightmark}
	\makeevenfoot{Ruled}{}{}{\pagenumberfont\thepage}
	\makeoddfoot{Ruled}{}{}{\pagenumberfont\thepage}
	\renewcommand*{\bibmark}{\markright{\bibname}} % I don't like empty headings!


\fi
\makeatother

%Change fonts: Page number sans-serif (much cleaner than the roman version)
\def\pagenumberfont{\sffamily}
% Change fonts: Section headers Sans-Serif:
\setsecheadstyle{\Large\sffamily\raggedright}
\setsubsecheadstyle{\large\sffamily\raggedright}
\setsubsubsecheadstyle{\normalsize\sffamily\raggedright}

% Title
\pretitle{\LARGE\sffamily}
\posttitle{\par\vspace{4ex}}
\preauthor{\large\sffamily\hspace{1cm}}
\postauthor{\par\vspace{3ex}}
\predate{\small\sffamily\hspace{1cm}Last Updated on: }
\postdate{\par\vspace{2cm}}
\copypagestyle{title}{plain}
\makeoddfoot{title}{}{}{}
\makeevenfoot{title}{}{}{}

\renewcommand{\abstractnamefont}{\sffamily}
\renewcommand{\abstracttextfont}{}
\renewcommand{\absnamepos}{flushleft}


\makechapterstyle{mestrado}{% Originally ``AlexanderGrebenkov'', adapted
\renewcommand{\chapterheadstart}{\goodbreak\vspace*{\beforechapskip}\medskip}
\renewcommand{\chapnamefont}{\normalfont\Large\scshape}
\renewcommand{\chapnumfont}{\normalfont\Large\scshape}
\renewcommand{\chaptitlefont}{\normalfont\Large\scshape}
\renewcommand{\printchaptername}{}
\renewcommand{\chapternamenum}{}
\renewcommand{\printchapternum}{\normalfont\Large\scshape\S\thechapter}
\renewcommand{\afterchapternum}{\hspace{1em}}
\renewcommand{\afterchaptertitle}{\par\nobreak\vspace{-.9em}\moveright 6pt\vbox to 1pt{\hrule width .64\textwidth}\nobreak\vskip\afterchapskip\nobreak}
}
\chapterstyle{mestrado}

\SetCommentSty{textsl}



\newcommand*{\fillunderscore}{~\hrulefill}
\newcommand*{\checkbox}{$\square$}
\newcommand*{\yesno}{\hfill\mbox{yes\checkbox~no\checkbox}}
\newcommand{\header}[1]{\textsl{#1}}
\def\th#1{#1}
\begin{document}
\maketitle
\begin{abstract}
Please fill out this survey. This will help me understand what was good and bad about this class. Try to be as precise and complete as possible.

The surveys are anonymous, but I make some results of it public.
\end{abstract}

\chapter{About You}

\header{This is just general information about you.}

Major/Programme: \fillunderscore\\
I am taking this class for credit: \yesno.

\chapter{Overall}

Did this class fulfill your expectations?
\begin{itemize}[\checkbox]
\item It surpassed them by a mile.
\item It surpassed them.
\item It fulfilled my expectations.
\item I was a bit disappointed.
\item It was a total waste of my time.
\end{itemize}

The goal of this class was to make you 20\% more productive. How would you estimate the productivity increase from the methods and tools you learnt?

\begin{itemize}[\checkbox]
\item Zero.
\item Just a bit, but less than 20\%.
\item Around 20\%.
\item More than 20\%.
\item More than twice as productive.
\end{itemize}

If you replied ``Zero'' or ``Just a bit,'' why do you think so:

\begin{itemize}[\checkbox]
\item I still haven't gone over the initial energy barrier, but I believe I will still get more productive with the tools I've learnt.
\item The tools don't help with the kind of problems I work on.
\item I didn't really understand what was being taught.
\end{itemize}

Or something else:\par
\fillunderscore\par
\fillunderscore\par
\fillunderscore\par
\fillunderscore\par

Here are the particular expectations that I had listed on the website. Let me know if you think that you fulfill them:

    (i) Software carpentry: source control, unit testing, profilers. Students should know how to use Subversion, nosetest, and the Python profiler as well as understand the concepts behind these tools (which will enable them to use them with a different implementation).

\begin{itemize}[\checkbox]
\item I understand this perfectly.
\item I understand the overall gist.
\item I have a vague idea.
\item I don't really get it.
\item What are you talking about?
\end{itemize}
Additional comments:
\fillunderscore\par
\fillunderscore\par
\fillunderscore\par


    (ii) Modern programming paradigms: object oriented. Students should know how polymorphism works and understand when and why it can be useful.
\begin{itemize}[\checkbox]
\item I understand this perfectly.
\item I understand the overall gist.
\item I have a vague idea.
\item I don't really get it.
\item What are you talking about?
\end{itemize}
Additional comments:
\fillunderscore\par
\fillunderscore\par
\fillunderscore\par


    (iii) Students should know how floating-point numbers are represented and their limitations. Detailed knowledge of specific formats is not required.
\begin{itemize}[\checkbox]
\item I understand this perfectly.
\item I understand the overall gist.
\item I have a vague idea.
\item I don't really get it.
\item What are you talking about?
\end{itemize}
Additional comments:
\fillunderscore\par
\fillunderscore\par
\fillunderscore\par


    (iv) Technologies: Python including numpy. Students should be able to confortably write medium-sized programs (a few thousand lines of code) using these technologies in an effective way.

\begin{itemize}[\checkbox]
\item I understand this perfectly.
\item I understand the overall gist.
\item I have a vague idea.
\item I don't really get it.
\item What are you talking about?
\end{itemize}
Additional comments:
\fillunderscore\par
\fillunderscore\par
\fillunderscore\par


\section{Particular Technologies \& Tools}

\header{These are the tools that were taught in this class. I would like to know if you think they were useful.}

\begin{tabular}[H]{lcccccccc}

\toprule
    & \th{Never} & \th{Sometimes} & \th{On Most Projects} & \th{Always}\\
\midrule
Python             &\checkbox &\checkbox &\checkbox &\checkbox \\
shell              &\checkbox &\checkbox &\checkbox &\checkbox \\
testing            &\checkbox &\checkbox &\checkbox &\checkbox \\
version control    &\checkbox &\checkbox &\checkbox &\checkbox \\
debugger           &\checkbox &\checkbox &\checkbox &\checkbox \\
                   &\checkbox &\checkbox &\checkbox &\checkbox \\
                   &\checkbox &\checkbox &\checkbox &\checkbox \\
                   &\checkbox &\checkbox &\checkbox &\checkbox \\
                   &\checkbox &\checkbox &\checkbox &\checkbox \\
                   &\checkbox &\checkbox &\checkbox &\checkbox \\
                   &\checkbox &\checkbox &\checkbox &\checkbox \\
                   &\checkbox &\checkbox &\checkbox &\checkbox \\
                   &\checkbox &\checkbox &\checkbox &\checkbox \\
\bottomrule
\end{tabular}


\chapter{Bugs}

\header{In this section, I ask what you believe was wrong with this class. It is one of the more important sections of the questionaire.}

\section{Missing}

What do you think was missing in this course? Feel free to mention particular technologies, techniques, or subjects.

\fillunderscore\par
\fillunderscore\par
\fillunderscore\par
\fillunderscore\par
\fillunderscore\par
\fillunderscore\par


\section{Too Much}

What did you think was too much in this class? Was there a subject/lecture that you didn't see the point of?

\fillunderscore\par
\fillunderscore\par
\fillunderscore\par
\fillunderscore\par
\fillunderscore\par
\fillunderscore\par

\section{Badly Explained}

Were there any topics you felt were particularly badly explained (hopefully not the whole course)?

\fillunderscore\par
\fillunderscore\par
\fillunderscore\par
\fillunderscore\par
\fillunderscore\par
\fillunderscore\par
\chapter{Comments}

\header{Please add any other comments you feel might be helpful.}

\fillunderscore\par
\fillunderscore\par
\fillunderscore\par
\fillunderscore\par
\fillunderscore\par
\fillunderscore\par
\fillunderscore\par
\fillunderscore\par
\fillunderscore

\end{document}
