\documentclass{beamer}
\usetheme{CMU}

\usepackage{pgf,pgfarrows,pgfnodes,pgfautomata,pgfheaps,pgfshade}
\usepackage{amsmath,amssymb}
\usepackage[utf8]{inputenc}
\usepackage{colortbl}
\usepackage[english]{babel}
\usepackage{booktabs}
\usepackage{slpython}
\usepackage{underscore}

\author{Luís Pedro Coelho}
\institute{Programming for Scientists}

\graphicspath{{figures/}{figures/generated/}{images/}}

\newcommand*{\code}[1]{\textsl{#1}}


\title{Programming for Scientists}
\date{Spring 2009}

\begin{document}

\frame{\maketitle}

\frame{\frametitle{Introduction}

\begin{block}{For Whom? What? Why?}
\begin{itemize}
\item For scientists who program.
\item An intermediate programming class.
\item To teach how to program \alert{better}.
\end{itemize}
\end{block}

\begin{block}{Goal}
A 20\% improvement in productivity.
\end{block}
}

\frame{\frametitle{Course Structure}
\begin{block}{Two Lectures a Week}
\begin{enumerate}
\item Tuesday session: a lecture where basic concepts are presented.
\item Tuesday session: a lab type session, where we discuss particular technologies that let us implement the basic concepts.
\end{enumerate}
\end{block}
}

\frame{\frametitle{Topics}

\begin{block}{Main Topics}
\begin{itemize}
\item Code organisation.
\item Software carpentry.
\item Numerics.
\item Software distribution.
\end{itemize}
\end{block}

\begin{block}{Main Technologies}
\begin{itemize}
\item Python programming language.
\item numpy \& scipy libraries.
\end{itemize}
\end{block}

}

\frame{\frametitle{More Information}

\begin{block}{Website}
http://coupland.cbi.cmu.edu/pfs

(Contains a detailed syllabus)
\end{block}

\begin{block}{Email}
lpc@cmu.edu
\end{block}
}

\end{document}
