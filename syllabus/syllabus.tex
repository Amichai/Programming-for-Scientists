\documentclass{article}

\title{Programming for Scientists}
\author{Lu\'\i s Pedro Coelho}

\newcommand*{\Lecture}[1]{%
{\bf #1}%
}

\begin{document}

\maketitle

\begin{abstract}
We propose a course which introduces students to programming for scientific computing. The target audience consists of working scientists with a need to program. We expect that they have some limited experience (perhaps only using simple languages like Matlab), but lack a formal introduction to programming. Students with no programming experience will be welcome and will be accommodated through extra lectures and lab sessions.

The course will consist of a weekly lecture (80~min) and a weekly lab session (80~min). The lab session will introduce specific technologies which implement the concepts described in the lecture. Students are expected to turn in weekly homeworks and a final project at the end. There will be no tests.

By the end of the course, students will understand the basic issues in programming, including principles of program design. They will be able to understand issues of precision and resource consumption. They should know of the basic algorithms for numerical optimisation and how to formulate problems as a numerical optimisation problem. In the lab sessions, they will have learned modern technologies with which to turn these concepts into a working program.
\end{abstract}

\section{Books}

Many books exist to introduce students to programming and the Python language used in the class. We recommend ``Dive into Python'' by Mark Pilgrim, which is available for free online (diveintopython.org) and in a paper edition (\$33 from Amazon, in Sept 2008; also available in the \textsc{cmu} library).

\section{Topics Covered}

\subsection{Module I: Introduction to Programming (4 weeks)}

In this module, we introduce students to code organisation paradigms. In the lab sessions, we introduce the Python language, which will be the basic technology used throughout the course.

\Lecture{Lecture 1.1}: Introduction

Overview of the course, course policies. Some examples of what students should be able to achieve by the end of the course.

\Lecture{Extra Lectures E1 \& Lab Session EL1} (for those who have no programming experience): Introduction to Programming. What is a type, what is a value, what is a name (variable).

\Lecture{Lab Session 1}: Introduction to Python. Basic types and iterations.

Homework: Compute the results of ``hash('your email')'' % This seemingly absurd assignment ensures that, by the second week, everyone has access to a working Python installation.

\Lecture{Lecture 1.2}: Code Organisation I (Procedural code). This lecture discusses how to organise your code into functions. Also discussed are documentation principles.

\Lecture{Lab Session 2}: Basic procedural constructs. How to define a function, module, package.

\Lecture{Lecture 1.3}: Code Organisation II (Introduction to Object Based Code). First element of object-oriented programming: using classes to organise related functions.

\Lecture{Lab Session 3}: Writing classes. The class statement, defining methods. Discussion of the importance of adhering to convention (apropos the \textsl{self} parameter).

\Lecture{Lecture 1.4}: Code Organisation III (Introduction to Object Oriented Code). Example of re-use of code. Basic principle of re-use.

\Lecture{Lab Session 4}: More advanced Python examples. Writing a class that inherits from another class. Example of a real script.

\subsection{Module II: Techniques for Scientific Computing (9 weeks)}

This is the main module of the course. We focus on both basic issues of scientific computing (floating point representation) and an overview of basic algorithms for scientific tasks (in particular, numerical optimisation).

\Lecture{Lecture 2.1}: How integers are represented in memory (positive and negative numbers), integer sizes. Thinking about memory allocation and temporaries.

\Lecture{Lab Session 5}: Introduction to numpy. Using arrays to write code, reductions and broadcasting.

\Lecture{Lecture 2.2}:  Floating point \textsc{i}. Discussiong of fixed-point vs.\ floating point. How floating point numbers are represented internally. \textsc{Ieee} numbers, nan's, Infs.

\Lecture{Lab Session 6}: Writing fast array-based code. 

\Lecture{Lecture 2.3}: Floating point II. Techniques to overcome under- and over-flow (using logarithms).

\Lecture{Lab Session 7}: Introduction to other scipy tools.

\Lecture{Lecture 2.4}: Optimisation as a programming tool I. Reformulating your problems as an optimisation problem. Limitations of this approach.

\Lecture{Lab Session 8}: Introduction to OpenOpt. Discussion of possible projects.

Homework: Students should submit a project proposal (or choose from the instructor proposed projects).

\Lecture{Lecture 2.5}: Optimisation as a programming tool II. Newton's method. Gradient descent.

\Lecture{Lab Session 9}: Implementation of a numerical algorithm.

\Lecture{Lecture 2.6}: Random processes. Pseudo-random numbers. Issues with stochasticity.

\Lecture{Lab Session 10}: Metropolis-Hastings Algorithm.

\Lecture{Lecture 2.7}: File parsing and regular expressions. Syntax of regular expressions.

\Lecture{Lab Session 11}: Parsing file formats.

\Lecture{Lecture 2.8}: Packaging your code for others. Publishing code is often part of the publication process with benefits for both the community and the author. In this lecture, we focus on the aspects inherent to a good, re-usable, software package

\Lecture{Lab Session 12}: setup.py. Discussion of open source distribution licenses and models.

\subsection{Module III: Advanced Topics (3 weeks)}

This final section consists of more advanced topics. No homeworks will be assigned as students should be working on their projects. The topics covered in this module is open to change based on student interests.

\Lecture{Lecture 3.1}: Graphical User Interfaces. Simple design principles behind an effective graphical user interface.

\Lecture{Lab Session 13}: Tools for building a user interface: pyqt.

\Lecture{Lecture 3.2}: Databases. Organising large quantities of data using a relational database.

\Lecture{Lab Session 14}: How to build a database.

\Lecture{Lecture 3.3}: Buffer time for overflow from other lectures.

\Lecture{Lab Session 15}: Tools for interface Python/C/C++/Fortran/R/\dots

\section{Grading}

Grading will be based on weekly homeworks and a final mini-project. There will be no exams.

The final project will be performed by students individually or in small groups (2 or~3 students). Students are encouraged to propose their own project topics (solving a scientific problem using techniques and technologies learned in this course), but instructors will also propose ideas.

\end{document}
